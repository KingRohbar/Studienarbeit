\chapter{Abschluss}\label{chpt:schluss}
Nach der Planung und Durchführung des Projektes wird das Projekt mit diesem Kapitel abgeschlossen. Es wird einerseits der Projektabschluss behandelt und andererseits ein Fazit und ein möglicher Ausblick über das Projekt abgegeben.

\section{Projektabschluss}\label{sec:projektschluss}
Das in der Projektdefinition definierte Ziel wurde erfolgreich umgesetzt und gilt als erreicht. Der gesetzte Zeitrahmen wurde ebenfalls eingehalten. Es konnten alle initial gesetzten Umfänge umgesetzt werden. Die genaueren Details sind den jeweiligen Abschnitten und Kapiteln dieser Studienarbeit zu entnehmen. Sie gilt als vollständige Dokumentation des Projekts.
In \autoref{sec:Ausblick} wird erläutert wie das Ergebnis des Projekts erweitert werden könnte. Das Projekt führte zu keinem technischen Fortschritt, hat jedoch seinen Zweck zum Sammeln von Erfahrung in der Android App Entwicklung in großem Umfang erfüllt. Spätere Projekte werden von der gesammelten Erfahrung durch erhöhte Effizienz und schnellere Fortschritte profitieren. Mit dem Ende dieser Arbeit gilt das Projekt als erfolgreich abgeschlossen.

\section{Fazit}\label{sec:fazit}
In diesem Abschnitt wird ein persönliches Fazit über das Projekt gezogen.\\
Das Projekt war von Beginn an sehr interessant für mich. Die Motivation hinter der Aufgabenstellung betrachtet ein Problem welches mir nicht fremd ist und ich mir daher gut vorstellen konnte die resultierende App selbst zu verwenden. Zu meinem Bedauern bin ich zum Start des Bearbeitungszeitraums in ein bekanntes Muster verfallen und habe mit dem Projektstart sehr lange gewartet. Dieses Projekt hat mir dieses Verhalten wieder deutlich gezeigt und mich daran erinnert vermehrt und intensiver an der Vermeidung dieser Verhaltensweise zu arbeiten. Nach dem Start des Projekts hat mir die Entwicklung viel Spaß bereitet und ich konnte im Verlauf des Projekts sehr viel dazulernen. Ich konnte alle aufgetretenen Probleme lösen und trotz verschiedener Herausforderungen zum ersten Mal eine eigene App entwickeln. Zudem konnte ich während der Entwicklung auf gelernte Techniken und Prinzipien aus unterschiedlichen Vorlesungen zurückgreifen und diese praktisch anwenden. Ich kann die Entwicklung einer App nur weiterempfehlen. Außerdem kann ich mir gut vorstellen privat weiter an der App zu Arbeiten, einerseits um sie zu verfeinern und eventuell die weiteren Ideen des Betreuers zu implementieren und um andererseits weitere speziellere Funktionen in der Android App Entwicklung kennenzulernen mit denen man bei der Arbeit für Rechner Anwendungen nicht in Kontakt kommt, \zB das Wischen oder die Nutzung der Sensoren an den Geräten.\\
Abschließend war das Projekt für mich eine Bereicherung.

\section{Ausblick}\label{sec:Ausblick}
Zum Abschluss der Studienarbeit und des Projekts wird ein Ausblick auf eine mögliche Fortführung des Projekts gegeben.\\
Die Anwendung befindet sich zum Abschluss des Projekts in dem, während der Planung gedachten, zufriedenstellenden Zustand. Falls das Projekt weitergeführt wird sollte zunächst durch weitere langfristigere Tests gewährleistet werden, dass keine größeren Fehler oder Probleme auftreten können. Der nächste Schritt wäre dann diese erste Version im Google PlayStore zu veröffentlichen. Im Anschluss daran könnte die Anwendung mit den nicht implementierten Ideen des Betreuers erweitert werden. Vor allem die Idee der Push-Benachrichtigungen würde eine gute Bereicherung für die App darstellen, da so auch das Vergessen des Abhakens vermieden werden könnte. Zusätzlich könnte die App mit Einstellungen, wie beispielsweise einem Darkmode oder Optionen für die Benachrichtigungen versehen werden.\\
Die App bietet sicher noch weitere Möglichkeiten in der Zukunft. Die hier genannten Schritte und Optionen beziehen sich auf die ersten Aufgaben welche im Anschluss an dieses Projekt gut umsetzbar und sinnvoll sind.