\chapter{Durchführung}\label{chpt:durchfuerung}

Dieses Kapitel beschreibt die Durchführung des Projekts. Explizit wird hier der Beginn und Verlauf der Entwicklung der \glqq Leave The House\grqq-App behandelt. Zu Beginn wird der Start der Projektdurchführung, gefolgt von der eigentlichen Entwicklung beschrieben. Im Anschluss daran wird das Testen der Applikation behandelt. Zum Schluss dieses Kapitels werden in der Problembehandlung alle in den Anderen Abschnitten aufgetretenen Probleme ausführlich behandelt.

\section{Projektdurchführung}\label{sec:projektdurchfuerung}

Die Projektdurchführung behandelt den Start der eigentlichen Arbeit des Projekts. Wie bereits in \autoref{chpt:planung} beschrieben ist das Risiko des verspäteten Projektstarts eingetreten. Dies führte dazu, dass die Arbeit an dem Projekt nicht wie geplant am 01.01.2020 gestartet wurde. Der Projektstart begann stattdessen im März 2021. Durch die in der \nameref{sec:risiko} festgelegte Alternative hatte der verspätete Start keine, abgesehen der aus der Alternative resultierenden, Auswirkungen auf die Durchführung.\\
Die Bearbeitung des Projekts begann im März 2021 und endete mit Erreichung des geplanten Umfang Mitte April 2021. Durch die bereits genannte Erweiterung des Umfang nach Rücksprache mit dem Betreuer wird der endgültige Projektabschluss auf das Abgabedatum der Studienarbeit, den 17.05.2021, verlegt. Das Ende der Bearbeitung des zuvor geplanten Umfang wird weiterhin als Mitte April festgehalten. In der Verlängerten Bearbeitungszeit wird versucht die weiteren Ideen für den Umfang des Betreuer zu implementieren, da in dieser Zeit auch die Studienarbeit geschrieben werden muss kann eine vollständige Dokumentation dieser in der Studienarbeit nicht gewährleistet werden.

\section{Entwicklung}\label{sec:entwicklung}
In diesem Abschnitt wird der vollständige Verlauf der Entwicklung behandelt. Von der Erstellung des Projekts bis hin zum ersten fertigen Zustand der App. Die während der Entwicklung aufgetretenen Probleme werden hier genannt, jedoch erst im \autoref{sec:problem} \nameref{sec:problem} ausführlich behandelt. Das erstellen der im Umfang festgelegten Tests wird ebenfalls in einem anderen Abschnitt (\ref{sec:tests} \nameref{sec:tests}) behandelt.

\subsection{Projekterstellung}\label{subsec:projekterstellung}
Zu Beginn der Arbeit an einem Projekt muss zuerst das Projekt erstellt werden. Dies hängt damit zusammen, dass \acp{IDE} in der Regel Projekte als oberste Ordnerstruktur nutzen. Alle Dateien innerhalb des Projektordners können somit diesem Projekt zugeordnet werden.\\
Die Entwicklung der App begann ebenfalls mit der Erstellung eines neuen Projekts in Android Studio. Dabei wird \glqq New $\rightarrow$ New Project\grqq in der File Dropdown-Liste ausgewählt. Da Android nicht nur als Betriebssystem für Smartphones und Tablets sondern auch für Smartwatches oder in Autos und TVs verwendet wird muss als erster Schritt angegeben werden für welche Endplattform man eine Anwendung entwickeln möchte. In diesem Schritt kann auch direkt eine Projektvorlage ausgewählt werden. Aufgrund von geringer Erfahrung in der App-Entwicklung wurde für dieses Projekt die Vorlage \glqq Basic Activity\grqq (Basis Aktivität) anstatt einer \glqq Empty Activity\grqq (Leere Aktivität) als Grundlage für das Projekt gewählt. Im Gegensatz zu einer leeren Aktivität Vorlage beinhaltet die Basic Activity bereits ein Einstellungsmenü in der Werkzeugleiste am oberen Bildschirmrand, einem Knopf in der unteren Rechten Bildschirmecke und einen Knopf in der Mitte des Bildschirms, der einen Anzeigenwechsel bewirkt. Diese bereits vorhandenen Elemente und Funktionen erleichtern den Einstieg in die Entwicklung, da der Entwickler sich den Code dieser ansehen und somit leichter die Funktionsweise und Aufbau von Android-Apps verstehen kann. Hilfreich hierbei ist der in der Android Studio \ac{IDE} eingebaute Emulator. Mit dem Emulator können virtuelle Android Geräte erstellt werden, um die App zu testen. Dabei kann die Android Version, so wie das Gerät ausgewählt werden. Der Emulator ist direkt zu Projektstart der Punkt an dem das erste Problem auftrat. Nach erstellen des Projekts kann man die gewählte Vorlage direkt über den Emulator testen. Dafür muss im AVD Manager ein neues Gerät erstellt werden, welches zum testen genutzt werden soll. Bis hierhin lief alles reibungslos. Beim ausführen der App kam jedoch dann das Problem zum Vorschein, der Emulator konnte nicht gestartet werden.
Es blieben nun nur zwei Möglichkeiten das Problem zu beheben, welche im \autoref{sec:problem} \nameref{sec:problem} behandelt werden. Trotzt des Problems war die Projekterstellung in Android Studio damit abgeschlossen. An diesem Punkt (nach Behebung des Problems) könnte mit der Entwicklung begonnen werden, doch ein wichtiges weit verbreitetes und empfohlenes Mittel kann noch zum Projekt hinzugefügt werden. Die Versionsverwaltung.\\
Mithilfe von Versionsverwaltung können Änderungen an Dateien erfasst und verwaltet werden. Ein übliches Versionsverwaltungssystem ist der kostenlose Dienst GitHub. In GitHub können Nutzer für Projekte sogenannte \glqq Repository\grqq, zu Deutsch Verwaltungsorte, anlegen. Innerhalb eines Repository werden die Projektdateien verwaltet. GitHub erfasst jede Änderung an einer Datei und kann diese dem Nutzer anzeigen. Mithilfe eines commit können die Änderungen dann als neue Version im Repository abgelegt werden. Dies erlaubt es mehreren Nutzern an der gleichen Datei zu arbeiten ohne sich ständig in die Quere zu kommen. Außerdem erlaubt es dem Nutzer immer wieder auf stabile Versionen zurückzugreifen falls die aktuellen Änderungen nicht zum gewollten Ergebnis geführt haben. Im Zuge dieses Projektes wurde auf GitHub ein neues Repository angelegt, welches als Versionsverwaltung für die App genutzt wird. Damit können beispielsweise erfolgreiche Implementierungen eines Anwendungsfalls mithilfe eines commits in GitHub gesichert werden.\\
Mit der Erstellung des GitHub Repository und Verknüpfung des Android Studio Projekts damit ist die Projekterstellung abgeschlossen. Alle Vorbereitungen sind somit getroffen worden um eine Erfolgreiche Entwicklung zu gewährleisten.


\subsection{Checklisten}\label{subsec:checklisten}

\subsection{Aufgaben}\label{subsec:aufgaben}

\subsection{Erweiterung}\label{subsec:erweiterung}

\section{Tests}\label{sec:tests}

\section{Problembehandlung}\label{sec:problem}