\chapter{Durchführung}\label{chpt:durchfuerung}

Dieses Kapitel beschreibt die Durchführung des Projekts. Explizit wird hier der Beginn und Verlauf der Entwicklung der \glqq Leave The House\grqq-App behandelt. Zu Beginn wird der Start der Projektdurchführung, gefolgt von der eigentlichen Entwicklung beschrieben. Im Anschluss daran wird das Testen der Applikation behandelt. Zum Schluss dieses Kapitels werden in der Problembehandlung alle in den Anderen Abschnitten aufgetretenen Probleme ausführlich behandelt.

\section{Projektdurchführung}\label{sec:projektdurchfuerung}

Die Projektdurchführung behandelt den Start der eigentlichen Arbeit des Projekts. Wie bereits in \autoref{chpt:planung} beschrieben ist das Risiko des verspäteten Projektstarts eingetreten. Dies führte dazu, dass die Arbeit an dem Projekt nicht wie geplant am 01.01.2020 gestartet wurde. Der Projektstart begann stattdessen im März 2021. Durch die in der \nameref{sec:risiko} festgelegte Alternative hatte der verspätete Start keine, abgesehen der aus der Alternative resultierenden, Auswirkungen auf die Durchführung.\\
Die Bearbeitung des Projekts begann im März 2021 und endete mit Erreichung des geplanten Umfang Mitte April 2021. Durch die bereits genannte Erweiterung des Umfang nach Rücksprache mit dem Betreuer wird der endgültige Projektabschluss auf das Abgabedatum der Studienarbeit, den 17.05.2021, verlegt. Das Ende der Bearbeitung des zuvor geplanten Umfang wird weiterhin als Mitte April festgehalten. In der Verlängerten Bearbeitungszeit wird versucht die weiteren Ideen für den Umfang des Betreuer zu implementieren, da in dieser Zeit auch die Studienarbeit geschrieben werden muss kann eine vollständige Dokumentation dieser in der Studienarbeit nicht gewährleistet werden.

\section{Entwicklung}\label{sec:entwicklung}
In diesem Abschnitt wird der vollständige Verlauf der Entwicklung behandelt. Von der Erstellung des Projekts bis hin zum ersten fertigen Zustand der App. Die während der Entwicklung aufgetretenen Probleme werden hier genannt, jedoch erst im \autoref{sec:problem} \nameref{sec:problem} ausführlich behandelt. Das erstellen der im Umfang festgelegten Tests wird ebenfalls in einem anderen Abschnitt (\ref{sec:tests} \nameref{sec:tests}) behandelt.

\subsection{Projekterstellung}\label{subsec:projekterstellung}
Zu Beginn der Arbeit an einem Projekt muss zuerst das Projekt erstellt werden. Dies hängt damit zusammen, dass \acp{IDE} in der Regel Projekte als oberste Ordnerstruktur nutzen. Alle Dateien innerhalb des Projektordners können somit diesem Projekt zugeordnet werden.\\
Die Entwicklung der App begann ebenfalls mit der Erstellung eines neuen Projekts in Android Studio. Dabei wird \glqq New $\rightarrow$ New Project\grqq{} in der File Dropdown-Liste ausgewählt. Da Android nicht nur als Betriebssystem für Smartphones und Tablets sondern auch für Smartwatches oder in Autos und TVs verwendet wird muss als erster Schritt angegeben werden für welche Endplattform man eine Anwendung entwickeln möchte. In diesem Schritt kann auch direkt eine Projektvorlage ausgewählt werden. Aufgrund von geringer Erfahrung in der App-Entwicklung wurde für dieses Projekt die Vorlage \glqq Basic Activity\grqq (Basis Aktivität) anstatt einer \glqq Empty Activity\grqq (Leere Aktivität) als Grundlage für das Projekt gewählt. Im Gegensatz zu einer leeren Aktivität Vorlage beinhaltet die Basic Activity bereits ein Einstellungsmenü in der Werkzeugleiste am oberen Bildschirmrand, einem Knopf in der unteren Rechten Bildschirmecke und einen Knopf in der Mitte des Bildschirms, der einen Anzeigenwechsel bewirkt. Diese bereits vorhandenen Elemente und Funktionen erleichtern den Einstieg in die Entwicklung, da der Entwickler sich den Code dieser ansehen und somit leichter die Funktionsweise und Aufbau von Android-Apps verstehen kann. Hilfreich hierbei ist der in der Android Studio \ac{IDE} eingebaute Emulator. Mit dem Emulator können virtuelle Android Geräte erstellt werden, um die App zu testen. Dabei kann die Android Version, so wie das Gerät ausgewählt werden. Der Emulator ist direkt zu Projektstart der Punkt an dem das erste Problem auftrat. Nach erstellen des Projekts kann man die gewählte Vorlage direkt über den Emulator testen. Dafür muss im AVD Manager ein neues Gerät erstellt werden, welches zum testen genutzt werden soll. Bis hierhin lief alles reibungslos. Beim ausführen der App kam jedoch dann das Problem zum Vorschein, der Emulator konnte nicht gestartet werden.
Es blieben nun nur zwei Möglichkeiten das Problem zu beheben, welche im \autoref{sec:problem} \nameref{sec:problem} behandelt werden. Nach wählen der Vorlage müssen noch weitere Informationen bei der Projekterstellung angegeben werden.
Einerseits muss der Name des Projekts angegeben werden, hier wurde der Name der Endgültigen App \glqq Leave The House\grqq{} eingegeben. Andererseits müssen grundlegende Entscheidung für die Entwicklung der App gefällt werden. Neben dem Namen muss noch die Programmiersprache und die minimal kompatible Anrdoid-Version gewählt werden. Android Apps können in zwei verschiedenen Sprachen geschrieben werden. Java und Kotlin. Für dieses Projekt wurde Kotlin als Programmiersprache für die App ausgewählt. Kotlin ist in der Android Entwicklung weit verbreitet und bietet eine Interoperabilität für Java. Somit kann während der App-Entwicklung auch auf Java zurückgegriffen werden, falls dies nötig sein sollte. Kotlin bietet im Gegensatz zu Java weitere Features wie null-Absicherung zurm Schutz vor NullPointer-Exceptions oder direkte View.Bindung.\footcite{Kotlin.2020} Als minimale Android-Version wurde die \ac{API} (Programmierschnittstelle) 23 festgelegt. Diese entspricht der Android Version 6.0 Marshmello und wird laut Android Studio zum aktuellen Zeitpunkt von 84.9\% der Geräte unterstützt. Nach diesen Angaben wurde die Erstellung des Projekts in Android Studi erfolgreich abgeschlossen. An diesem Punkt (nach Behebung des Problems) könnte mit der Entwicklung begonnen werden, doch ein wichtiges weit verbreitetes und empfohlenes Mittel kann noch zum Projekt hinzugefügt werden. Die Versionsverwaltung.\\
Mithilfe von Versionsverwaltung können Änderungen an Dateien erfasst und verwaltet werden. Ein übliches Versionsverwaltungssystem ist der kostenlose Dienst GitHub. In GitHub können Nutzer für Projekte sogenannte \glqq Repository\grqq, zu Deutsch Verwaltungsorte, anlegen. Innerhalb eines Repository werden die Projektdateien verwaltet. GitHub erfasst jede Änderung an einer Datei und kann diese dem Nutzer anzeigen. Mithilfe eines commit können die Änderungen dann als neue Version im Repository abgelegt werden. Dies erlaubt es mehreren Nutzern an der gleichen Datei zu arbeiten ohne sich ständig in die Quere zu kommen. Außerdem erlaubt es dem Nutzer immer wieder auf stabile Versionen zurückzugreifen falls die aktuellen Änderungen nicht zum gewollten Ergebnis geführt haben. Im Zuge dieses Projektes wurde auf GitHub ein neues Repository angelegt, welches als Versionsverwaltung für die App genutzt wird. Damit können beispielsweise erfolgreiche Implementierungen eines Anwendungsfalls mithilfe eines commits in GitHub gesichert werden.\\
Mit der Erstellung des GitHub Repository und Verknüpfung des Android Studio Projekts damit ist die Projekterstellung abgeschlossen. Alle Vorbereitungen sind somit getroffen worden um eine Erfolgreiche Entwicklung zu gewährleisten.


\subsection{Erstellen der Checkliste}\label{subsec:erstelleChecklisten}

Nach der erfolgreichen Projekterstellung begann die Arbeit an dem Projekt mit der Realisierung des ersten Anwendungsfall, dem erstellen einer Checkliste. Dafür wurde die Klasse Checkliste angelegt, diese wird in \autoref{code:checkliste} gezeigt. Die Klasse besteht aus einem initialen Konstruktor, welcher den Titel und die Beschreibung, sowie einem zweiten Konstruktor der neben Titel und Beschreibung noch eine Liste von Aufgaben entgegennimmt. Zudem enthält die Klasse, wie in \autoref{sec:umfang} beschrieben, die Methoden um ein Element der Aufgaben Liste hinzuzufügen und um ein Element aus der Liste zu entfernen. Die ebenfalls beschrieben get() und set() Methoden sind in dieser Klasse nicht vorhanden, da Kotlin diese Methoden standardmäßig durch Zugriff auf die Variable zur Verfügung stellt. Um also auf den Titel einer Checkliste zuzugreifen genügt $checklist.title$ anstelle von $checklist.getTitle()$. Nachdem das Modell zum halten der Checkliste, die Checkliste-Klasse erstellt wurde muss nun die Funktion implementiert werden ein Checklisten Objekt zu erstellen und auf dem Bildschirm anzuzeigen.
\\
\lstinputlisting[
label=code:checkliste,    % Label; genutzt für Referenzen auf dieses Code-Beispiel
caption=Checkliste Klasse,
captionpos=b,               % Position, an der die Caption angezeigt wird t(op) oder b(ottom)
style=EigenerKotlinStyle,     % Eigener Style der vor dem Dokument festgelegt wurde
firstline=1,                % Zeilennummer im Dokument welche als erste angezeigt wird
lastline=14                 % Letzte Zeile welche ins LaTeX Dokument übernommen wird
]{Quellcode/Checkliste.kt}

Bevor diese Funktion jedoch implementiert werden kann muss der Ablauf dafür festgelegt werden. Der Nutzer soll auf den in der rechten unteren Bildschirmecke befindlichen Knopf drücken um eine neue Ansicht zu öffnen. Diese zeigt Eingabefelder für den Titel und die Beschreibung in denen der Benutzer diese Angaben tätigt, welche dann für den Konstruktor der Checklisten Klasse genutzt werden. Über einen Knopf an der gleichen Position wie der vorherige wird die Eingabe bestätigt und die Checkliste mit den Eingegeben Werten erstellt. Die erstellte Checkliste soll dann als Element einer Liste auf dem Bildschirm dargestellt werden. Anhand dieses Ablaufs wurde dann die Implementation dieser Funktion und Anwendungsfall begonnen.\\
Android besteht Grundlegend aus Aktivitäten. Die Grundaktivität ist die sogenannte \glqq Main Activity\grqq. Diese bildet den Einstiegspunkt in die App und wird beim öffnen der App angezeigt. Eine Aktivität besteht meistens aus einer \grqq Controller-\grqq{} und einer Layout Datei. In der Layout-Datei wird definiert was auf dem Bildschirm angezeigt wird wenn die Aktivität ausgeführt wird. Dabei handelt es sich um eine XML-Datei in der Elemente wie Knöpfe, Text oder weitere mithilfe eines Layouts angeordnet werden können. Die Controller-Datei spiegelt dagegen die funktionale Ebene der Aktivität wieder. In ihr werden Funktionen ausgeführt und der Aktivitäts-Lebenszyklus behandelt. Eine Aktivität hat einen Lebenszyklus der aus verschiedenen Zuständen besteht. Der erste Zustand der ausgeführt wird ist \glqq onCreate()\grqq. In dieser Methode wird angegeben welches Layout dem Nutzer angezeigt werden soll. Diese Methode wird in der Regel immer überschrieben, um mittels setContentView() das Layout anzugeben. \autoref{code:onCreate} zeigt die onCreate-Methode der Aktivität CreateChecklist. Hier wird wie angegeben die onCreate-Methode überschrieben und über setContentView das zugehörige Layout zur Darstellung auf dem Bildschirm angegeben. Zudem wird über die seSupportActionBar-Methode die im Layout definierte Werkzeugleiste als SupportActionBar festgelegt. Damit wird der Aktivität ermöglicht die Interaktion mit eventuell in der Werkzeugleiste vorhanden Knöpfen zu erfassen und entsprechende Funktionen auszuführen. In der weiteren Beschreibung zur Implementierung des Anwendungsfall zum erstellen einer Aktivität wird weiter auf das Codebeispiel eingegangen. Die weiteren Zustände des Lebenszyklus sind onStart() (Startet die Aktivität und macht sie sichtbar und ermöglicht Interaktivität), onResume() (Wird ausgeführt wenn die Aktivität wieder Interagierbar wird), onPause() (Dieser Zustand tritt auf wenn die Aktiviät den Fokus verliert und ist oft ein Indikator dass die Aktivität verlassen wird), onStop() (Die Aktivität ist nicht länger sichtbar für den Nutzer), onRestart(Wechsel von onStop() in onStart()) und onDestroy() (Zerstört die Aktivität und endet den Lebenszyklus).\footcite{Aktivitäten.2021} Die Grundlegende Funktionsweise von Aktivitäten sollte nun verstanden sein.
\\
\lstinputlisting[
label=code:onCreate,    % Label; genutzt für Referenzen auf dieses Code-Beispiel
caption=onCreate Methode der CreateChecklist Aktivität,
captionpos=b,               % Position, an der die Caption angezeigt wird t(op) oder b(ottom)
style=EigenerKotlinStyle,     % Eigener Style der vor dem Dokument festgelegt wurde
firstline=1,                % Zeilennummer im Dokument welche als erste angezeigt wird
lastline=21                 % Letzte Zeile welche ins LaTeX Dokument übernommen wird
]{Quellcode/onCreate.kt}

Um den Anwendungsfall eine Checkliste erstellen zu realisieren wird zunächst eine neue Aktivität erstellt. Dazu wird über File $\rightarrow$ New $\rightarrow$ Activity $\rightarrow$ Empty Activity eine neue Aktivität angelegt. Hierbei muss der Name der Aktivität und des Layout angegeben werden. Der Name des Layout wird von Android Studio passend zu der Eingabe im Aktivitätsnamen-Feld angepasst. Auch hier kann die Programmiersprache gewählt werden. Das lässt sich auf die Interoperabilität zurückführen, da damit auch eine Java-Aktivität in einer Kotlin Anwendung ausgeführt werden kann.

\subsection{Speichern von Checklisten}\label{subsec:speichereCheckliste}

\subsection{Erstellen von Aufgaben}\label{subsec:erstelleAufgaben}

\subsection{Erweiterung}\label{subsec:erweiterung}

\section{Tests}\label{sec:tests}

\section{Problembehandlung}\label{sec:problem}