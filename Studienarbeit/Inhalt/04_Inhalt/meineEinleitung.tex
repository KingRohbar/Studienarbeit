\chapter{Einführung}\label{chpt:einführung}

Diese Arbeit beschreibt die Planung und Durchführung des Projekts \glqq Leave The House - Checklisten App\grqq im Rahmen der Studienarbeit der \ac{DHBW} Karlsruhe.\\
In diesem Kapitel werden die Motivation hinter der Anwendung so wie die vor Beginn der Arbeit festgelegte Aufgabenstellung beschrieben. Im weiteren Verlauf der Arbeit wird tiefer auf die Planungs- sowie die Durchführungsphase eingegangen und aufgetretene Probleme erläutert. Zum Schluss wird ein Fazit über die Gesamtheit des Projekts abgegeben und ein möglicher Ausblick zur Weiterführung beschrieben.

\section{Motivation}\label{sec:motivation}

Das altbekannte Problem: Man ist zu spät dran und muss dringend los, doch im Hinterkopf kommt immer wieder der Gedanke was vergessen zu haben... Sind die Fenster geschlossen, der Ofen abgestellt, ist alles eingepackt? Solche oder andere Aufgaben schwirren einem dann durch den Kopf. Hat man es endlich geschafft das Haus zu verlassen und steht vor dem Auto oder Fahrrad kommt wieder so ein Gedankenblitz, habe ich jetzt in all der Eile die Tür überhaupt abgeschlossen?\\
Solche Erfahrungen habe ich in der Zeit meines Studiums selbst häufiger erlebt und habe zu oft einen extra Weg zurück zur Tür meines Wohnheimzimmers gemacht nur um fest zu stellen, dass die Tür in den häufigsten fällen doch abgeschlossen ist. Die \glqq Leave The House - Checklisten App\grqq{} soll diesem Problem der Unsicherheit Abhilfe verschaffen. In dieser Mobile-App sollen Checklisten für jegliche Situationen in denen das Haus verlassen wird angelegt werden können. Die Checklisten enthalten jeweils alle Aufgaben die vor oder beim Verlassen des Hauses erledigt werden müssen. Sollte es wieder vorkommen, dass ein Gedanken des vergessen Habens vor dem Auto aufkommt, kann einfach in der App überprüft werden ob die Aufgabe erledigt wurde und sich so ein unnötiger Weg und das Gefühl etwas vergessen zu haben gespart werden.

\section{Aufgabenstellung}\label{sec:Aufgabenstellung}

Die Aufgabenstellung spiegelt den Kerngedanken hinter der App wieder.\\
Es soll eine iOS oder Android-App entwickelt werden.
In dieser sollen anpassbare Checklisten angelegt werden können, die einen beim verlassen des Haus oder Arbeitsplatz unterstützen. Somit sollen Aufgaben die immer oder meistens beim verlassen eines Ortes auftreten in einer Checkliste erfasst und als erledigt markiert werden können.\\
Im folgenden \autoref{chpt:planung} \nameref{chpt:planung} werden die in \autoref{sec:motivation} \nameref{sec:motivation} genannten Anreize und die hier festgelegte Aufgabenstellung aufgegriffen und die daraus resultierte Projektplanung beschrieben.